\documentclass[12pt, a4paper]{article}

\usepackage[utf8]{inputenc}
\usepackage[T1]{fontenc}
\usepackage[russian]{babel}
\usepackage[oglav,spisok,boldsect,eqwhole,figwhole,hyperref,hyperprint,remarks,greekit]{./style/fn2kursstyle}

\graphicspath{{./style/}{./figures/}}
\usepackage{./style/freetikz}
\usepackage{multirow}
\usepackage{supertabular}
\usepackage{multicol}
\usepackage{float}
\usepackage{amsthm}
%\usepackage{./style/mmacells}
%\usepackage{./style/mma}

% Параметры титульного листа
\title{HLL, HLLC и HLLD схемы для решения задач магнитной гидродинамики}
\author{И.\,П.~Шаманов}
\supervisor{В.\,В.~Лукин}
\group{ФН2-61Б}
\date{2024}

\begin{document}
	
	\maketitle
	
	\tableofcontents
	
	\newpage
	
	\section-{Введение}
	 В~физике и~технике возникают проблемы, связанные с~необходимостью изучения движения сильно нагретых ионизированных жидкостей и~газов в~электромагнитом поле. Такие задачи ставят, например, астрофизика, аэродинамика больших скоростей и~управляемый термоядерный синтез. В~связи с~этим в~настоящее время одной из наиболее интенсивно развивающихся областей механики и~физики является магнитная гидродинамика (МГД), которая описывает новые механические эффекты и~открывает новые методы воздействия на движение электропроводящих жидкостей и~газов в~электромагнитном поле.
	 
	 Для описания магнитогидродинамической среды представлены базовые уравнения, непосредственно аналитическое исследование которых возможно лишь в ограниченном ряде случаев. Однако подобные точные решения позволяют получить лишь качественные особенности поведения объекта. В инженерных же приложениях одной из самой главных задач является получение точных количественных результатов. Поэтому возникает потребность в робастных численных методах для МГД-задач.
	 
	 В данной работе представлены 3 схемы решения проблем идеальной МГД: Хартена---Лакса---Лира (HLL), HLLC и HLLD. Отличительной особенностью, объединяющей данные методы, является уход от линеаризации поставленной задачи. Поэтому расчёты по схемам обладают высоким разрешением. 
	 
	
	 \pagebreak
	
	\section{Постановка задачи}
	Опишем идеальную плазму. Выберем декартову прямоугольную систему координат. Пусть объём $V$ (ограниченная область), окружённый поверхностью $S$ с внешней единичной нормалью $\bi{n}$, помещён в электромагнитное поле и содержит в себе ионизированный газ.  
	
	Для описания электромагнитного поля будем использовать уравнения Максвелла --- обобщение опытных фактов, представляющее фундаментальные постулаты --- в~дифференциальной форме. Будем использовать гауссову систему единиц, в которой для напряжённости магнитного поля $\bi{H}$ и магнитной индукции $\bi{B}$ выполнено $\bi{B} = \mu \bi{H}$. Для полностью ионизированной плазмы положим магнитную проницаемость $\mu = 1$. Система уравнений имеет вид \cite{Kulikovskiy}
	\begin{equation}
	\label{Maxwell}
	\begin{cases}
		\nabla \times \bi{B} = \dfrac{4\pi}{c}\bi{j} + \dfrac1c \dfrac{\partial \bi{E}}{\partial t}, \\[3mm]
		\nabla \times \bi{E} =  -\dfrac1c \dfrac{\partial \bi{B}}{\partial t}, \\[3mm]
		\nabla \cdot \bi{E} = 4\pi\rho_{e}, \\
		\nabla \cdot \bi{B} = 0,
	\end{cases}
	\end{equation}
	где $\bi{E}$ --- напряжённость электрического поля, $\bi{j}$ --- плотность электрического тока, $c$~--- скорость света в вакууме, $\rho_e$ --- плотность заряда.
	
	Для описания сплошной среды будем использовать лагранжев подход, в котором прослеживается изменение во времени величин в первоначально зафиксированном объёме. Рассмотрим подсистему, состоящую из уравнений сохранения массы, импульса и энергии в~присутствии электромагнитных сил. Общая форма таких уравнений 
	\begin{equation}
	\label{Movement}
	\begin{cases}
	\displaystyle	\dfrac{d}{dt} \int_V \rho dV = 0, \\[3mm]
	\displaystyle	\dfrac{d}{dt} \int_V \rho\bi{v} dV = \oint_S \bi{p}_n dS + \int_V \bi{f} dV, \\[3mm]
	\displaystyle \dfrac{d}{dt} \int_V \rho \left(\varepsilon + \frac{\bi{v}^2}{2}\right) dV = -\int_S q_n dS + \int_S \bi{p}_n \cdot \bi{v} dS + \int_V A dV,
	\end{cases}
	\end{equation}
	где $\rho$ --- плотность среды, $\frac{d}{dt}$ --- субстанциональная производная, $\bi{v}$ --- скорость среды, $\varepsilon$ --- удельная внутренняя энергия, $\bi{p}_n$ --- плотность поверхностных сил, $\bi{f}$ --- плотность внешних объёмных сил, $q_n$ --- поток тепла через $S$, $A$ --- разность полного притока энергии к выделенному объёму и притока энергии за счёт потока тепла через его поверхность и работы поверхностных сил.
	
	Будем рассматривать только силы, действующие со стороны электромагнитного поля на среду, то есть силу Лоренца
	\[
	\bi{f} = \rho_e\bi{E} + \dfrac{1}{c}(\bi{j} \times \bi{B}).
	\]
	Из уравнений Максвелла \eqref{Maxwell} получим
	\[
		\bi{f} = - \dfrac{\partial \bi{g}}{\partial t} + \nabla \cdot \bi{\hat{T}},
	\]
	где $\bi{g} = \dfrac{1}{4\pi c}(\bi{E}\times\bi{B})$, $\bi{\hat{T}}: T_{ij} = \dfrac{1}{4\pi}(B_iB_j + E_iE_j)-\dfrac{1}{8\pi}(B^2 + E^2)\delta_{ij}$. \\[-0.5mm]
	
	Полный приток энергии, который приносится электромагнитным полем в единичный объём за единицу времени
	\[
	A = \bi{j} \cdot \bi{E}.
	\]
	
	Введём тензор напряжений $\bi{\hat{P}}: \hat{P}_{ij} = p_{ij} = p_{ji}$, где $p_{ij}$ есть напряжение в среде в~направлении орта $\bi{e}_i$ на площадке с нормалью $\bi{e}_j$. Соответственно, введя вектор $\bi{p}_j = p_{ij}\bi{e}_{i}$, получим полную силу, действующую на единичный элемент площади $S$ с~нормалью $\bi{n}$
	\[
	\bi{p}_n = \bi{p}_j n_j.
	\]
	Если рассматривать течения идеального газа в отсутствии вязкости и теплопроводности, то
	\[
		\bi{\hat{P}} = -p \bi{\hat{I}},
	\] 
	где \bi{\hat{I}} --- единичный тензор.
	
	Далее применим формулу Остроградского---Гаусса к поверхностным интегралам, заменим субстанциональные производные $\frac{d}{dt}$ на эйлеровы частные производные $\frac{\partial}{\partial t}$ и~примем во внимание тот факт, что $V$ --- произвольный объём. Тогда систему уравнений \eqref{Movement} можно записать в форме законов сохранения
		\begin{equation}
		\label{Movement2}
		\begin{cases}
		\dfrac{\partial\rho}{\partial t} + \nabla \cdot (\rho \bi{v})   = 0, \\[3mm]
		\dfrac{\partial}{\partial t}(\rho \bi{v} + \bi{g}) + \nabla\cdot\bi{\hat{\Pi}} = 0, \\[3mm]
		\dfrac{\partial e}{\partial t} + \nabla\cdot\bi{S} = 0,
		\end{cases}
	\end{equation}
	где $\bi{\hat{\Pi}}: \Pi_{ij} = \rho v_i v_j + p\delta_{ij} - T_{ij}$, $e = \rho(\varepsilon + \sfrac{\bi{v}^2}{2})+\sfrac{E^2+B^2}{8\pi}$, $\bi{S} = (\rho(\varepsilon + \sfrac{\bi{v}^2}{2})+ p)\bi{v} + \sfrac{c}{4\pi}(\bi{E}\times\bi{B})$.
	
	Используем закон Ома для плотности электрического тока в виде суммы плотности токов проводимости и конвективного
	\[
	\bi{j} = \sigma \left(\bi{E}+ \dfrac1c \bi{v}\times \bi{B}\right) + \rho_e\bi{v},
	\]
	где $\sigma$ есть коэффициент проводимости.
	
	Предположим \cite{Kulikovskiy}, что проводимость настолько велика, что выполнено
	\[
	\dfrac{1}{\sigma t_*} \ll 1, \phantom{xxx} \dfrac{v_*}{\sigma L} \ll 1,
	\]
	где $t_*$, $v_*$ и $L$ --- характерные время, скорость и длина. Тогда в первом уравнении системы \eqref{Maxwell} можно пренебречь членами $\sfrac{\partial \bi{E}}{\partial t}$ и $4\pi\rho_e\bi{v}$ по сравнению с $4\pi\sigma\bi{E}$. Исключим напряжённость электрического поля из системы \eqref{Maxwell}, \eqref{Movement2}
	\[
	\bi{E} = \dfrac1c \left(\dfrac{c^2}{4\pi\sigma}\nabla\times\bi{B} - \bi{v}\times\bi{B}\right).
	\]
	Тогда получим 
	\[
	\dfrac{\partial \bi{B}}{\partial t} = \nabla \times (\bi{v}\times\bi{B}) - \nabla \times (\dfrac{c^2}{4\pi\sigma} \nabla \times \bi{B}).
	\]
	Если магнитное число Рейнольдса $Re_m = \sfrac{4\pi\sigma v_* L}{c^2} \gg 1$, то пренебрегаем вторым членом и получаем
	\[
		\dfrac{\partial \bi{B}}{\partial t} = \nabla \times (\bi{v}\times\bi{B}).
	\]
	
	Применив соотношение $\nabla \times ( \bi{v}\times\bi{B} ) = \nabla\cdot(\bi{B}\otimes\bi{v}-\bi{v}\otimes\bi{B})$, сведём систему \eqref{Maxwell} к уравнению Фарадея и уравнению отсутствия магнитного заряда:
	\[
	\begin{cases}
		\dfrac{\partial \bi{B}}{\partial t} + \nabla \cdot (\bi{v}\otimes\bi{B}-\bi{B}\otimes\bi{v}) = 0, \\
		\nabla\cdot\bi{B} = 0.
	\end{cases}
	\]
	
	Будем рассматривать нерелятивистский случай при $\sfrac{v_*^2}{c^2}\ll1$, поэтому токами смещения $\sfrac{1}{4\pi}\sfrac{\partial\bi{E}}{\partial t}$ и конвективными токами $\rho_e \bi{v}$ пренебрежём по сравнению с полным током $c\nabla\times\bi{B}/4\pi$, поэтому
	\[
	\bi{j} = \sigma \left(\bi{E}+ \dfrac1c \bi{v}\times \bi{B}\right),
	\]
	\[
	\nabla\times\bi{B} = \dfrac{4\pi}{c}\bi{j}.
	\]
	
	Отметим, что в поставленных предположениях $\bi{B}$ и $\bi{j}$ остаются неизменными в~любой инерциальной системе координат, так как электромагнитное преобразуются при смене неподвижной системы координат (переменные без штриха) на систему координат, движущуюся со скоростью $\bi{v}$ (переменные со штрихом) следующим образом:
	\[
		\bi{B}' = \bi{B}-\dfrac{1}{c}(\bi{v}\times\bi{E}),
	\]
	\[
	\bi{j}' = \bi{j} - \rho_e \bi{v}.
	\]
	
	Воспользуемся и формулой $(\bi{v}\times\bi{B})\times\bi{B} = -\bi{v}B^2 + \bi{B}(\bi{v}\cdot\bi{B})$.
	\pagebreak
	
	Таким образом, итоговая система МГД-уравнений в дивергентной форме
	\begin{equation}
	\label{MHD-system}
	 \begin{cases}
	 	\dfrac{\partial\rho}{\partial t} + \nabla \cdot (\rho \bi{v})   = 0, \\[3mm]
	 	\dfrac{\partial \rho \bi{v}}{\partial t} + \nabla\cdot\left[\rho\bi{v}\otimes\bi{v} + p_T \bi{\hat{I}} - \dfrac{\bi{B}\otimes\bi{B}}{4\pi}\right] = 0, \\[3mm]
	 	\dfrac{\partial e}{\partial t} + \nabla\cdot \left[(e+p_T)\bi{v} - \dfrac{\bi{B}(\bi{v}\cdot\bi{B})}{4\pi}\right] = 0, \\[3mm]
	 		\dfrac{\partial \bi{B}}{\partial t} + \nabla \cdot (\bi{v}\otimes\bi{B}-\bi{B}\otimes\bi{v}) = 0,
	 \end{cases}
	\end{equation} 
	где $e = \rho(\varepsilon + \sfrac{\bi{v}^2}{2})+\sfrac{B^2}{8\pi}$ --- полная энергия единицы объёма газа, $p_T = p + \sfrac{B^2}{8\pi}$ --- полное давление.
	
	Заметим, что условие $\nabla \cdot \bi{B} =0$ выполнено, так как, применив оператор дивергенции к уравнению Фарадея, получим $\sfrac{\partial}{\partial t} (\nabla \cdot \bi{B}) = 0$. Поэтому, если при $t=0$: $\nabla\cdot\bi{B}=0$, то $\nabla\cdot\bi{B}\equiv0$.
	
	Далее будем рассматривать одномерный случай уравнений \eqref{MHD-system}. Обозначим компоненты векторов $\bi{v} = (u, v, w)^T$, $\bi{B} = (B_x, B_y, B_z)^T$. Запишем систему в векторном виде 
	\begin{equation}
	\label{vectorMHD}
	\dfrac{\partial \bi{U}}{\partial t} + \dfrac{\partial \bi{F}}{\partial x} = \bi{0},
	\end{equation}
	приняв
	\[
	\bi{U} = \begin{pmatrix}\rho \\ \rho u \\ \rho v \\ \rho w \\ e \\ B_x \\ B_y \\ B_z \end{pmatrix},
	\bi{F}=\begin{pmatrix} \rho u \\ 
					\rho u^2 + p_T - \sfrac{B_x^2}{4\pi} \\[3mm] 
					\rho uv - \sfrac{B_x B_y}{4\pi} \\[3mm] 
					\rho uw - \sfrac{B_x B_z}{4\pi} \\[3mm]
					(e+p_T)u - \sfrac{B_x }{4\pi}(\bi{v}\cdot\bi{B}) \\[3mm]
					0\\
					uB_y - vB_x \\[3mm]
					uB_z - wB_x  
			\end{pmatrix}.
	\]
	\pagebreak
	
	Для упрощения дальнейших выкладок перейдём в новую систему единиц, домножив компоненты вектора магнитной индукции на $\sqrt{4\pi}$. Заметим, что $B_x = const$. Получим следующую систему
 	\begin{equation}
 		\label{vectorMHD-2}
 		\dfrac{\partial \bi{U}}{\partial t} + \dfrac{\partial \bi{F}}{\partial x} = \bi{0},
 	\end{equation}
 	где 
 	\[
 	\bi{U} = \begin{pmatrix}\rho \\ \rho u \\ \rho v \\ \rho w \\ e \\ B_y \\ B_z \end{pmatrix},
 	\bi{F}=\begin{pmatrix} \rho u \\ 
 		\rho u^2 + p_T - B_x^2\\[3mm] 
 		\rho uv - B_x B_y \\[3mm] 
 		\rho uw - B_x B_z \\[3mm]
 		(e+p_T)u - B_x  (\bi{v}\cdot\bi{B}) \\[3mm]
 		uB_y - vB_x \\[3mm]
 		uB_z - wB_x  
 	\end{pmatrix}.
 	\]
 	
 	В качестве давления выберем $p=(\gamma-1)\left(e - \sfrac{1}{2}\rho\bi{v}^2 - \sfrac12 \bi{B}^2\right)$, где $\gamma$ --- показатель адиабаты. Тогда $p_T = p + \sfrac12 \bi{B}^2$.
 	
 	Собственные значения якобиана $A = \sfrac{\partial \bi{F}}{\partial \bi{U}}$ действительны:
 	\[
 	\lambda_{2,6} = u \mp c_a, \phantom{xx}\lambda_{1,7} = u \mp c_f,\phantom{xx} \lambda_{3,5} = u \mp c_s,\phantom{xx} \lambda_{4} = u,    
 	\]
 	где 
 	\[
 	c_a = \dfrac{|B_x|}{\sqrt{\rho}}, \phantom{xxx} c_{f,s} = \left\{
 			\dfrac{\gamma p + \bi{B}^2 \pm \sqrt{\left(\gamma p + \bi{B}^2\right)^2 - 4\gamma p B_x^2}}{2\rho}\right\}^{\sfrac12}.
 	\]
 	
 	Выполняются равенства
 	\[
 	\lambda_1 \le \lambda_2 \le \lambda_3 \le \lambda_4 \le \lambda_5 \le \lambda_6 \le \lambda_7,
 	\]
 	поэтому некоторые собственные значения могут совпадать в зависимости от направления и модуля вектора напряжённости магнитного поля.  Следовательно, МГД-уравнения не являются строго гиперболическими. Более того, как показали M. Brio и C. Wu (1988), вектор-функция потока $\bi{F}(\bi{U})$ не является выпуклой. Поэтому решение задачи Римана для МГД-системы может содержать в себе помимо обычных ударных волн и волн разрежения ещё и другие волны, например, составные и ударные волны сжатия.
 	
 	\section{Решение задачи}
 	\subsection{Метод HLL}
 	
 	Проинтегрируем уравнение  \eqref{vectorMHD-2} по области $(x_{i-1},x_{i})$: $x_{i}-x_{i-1}=h$, $i=\overline{1,n}$%, положив, что каждая компонента вектор-функций $\bi{U}$ и $\bi{F}$ является постоянной на $(x_{i-1},x_{i})$%
 	\[
 		\displaystyle \int_{x_{i-1}}^{x_i} \left(\dfrac{\partial \bi{U}}{\partial t}+\dfrac{\partial \bi{F}}{\partial x}\right)dx = 0;
 	\]
 	\[
 		\dfrac{\partial }{\partial t}\dfrac1h\displaystyle \int_{x_{i-1}}^{x_i} \bi{U}dx +
 		 \dfrac1h \left(\bi{F}|_{x_i}-\bi{F}|_{x_{i-1}}\right)= 0.
 	\]
 	
 	Далее будем рассматривать усреднённые значения $\langle\bi{U}\rangle_i = \sfrac1h \displaystyle\int_{x_{i-1}}^{x_i}\bi{U}dx$ или примем, что компоненты $\bi{U}$ являются кусочно-постоянными на $(x_{i-1},x_{i})$, $i=\overline{1,n}$: 
 	\begin{equation}
 		\label{afterxint}
 		\dfrac{\partial }{\partial t}\displaystyle \bi{U}_i +
 	\dfrac1h \left(\bi{F}|_{x_i}-\bi{F}|_{x_{i-1}}\right)= 0.
 	\end{equation}
 
 	Проинтегрируем \eqref{afterxint} по времени $(t_j,t_{j+1})$: $t_{j+1}-t_{j}=\tau$
 	\[
 	\displaystyle\int_{t_j}^{t_{j+1}}\dfrac{\partial }{\partial t}\bi{U}_i dt +  \dfrac1h	\displaystyle\int_{t_j}^{t_{j+1}}\left(\bi{F}|_{x_i}-\bi{F}|_{x_{i-1}}\right)dt = 0; 
 	\]
 	\[
 	\dfrac{\bi{U}^{j+1}_i-\bi{U}^{j}_i}{\tau} + \dfrac1h \left[
 	\dfrac1\tau \displaystyle\int_{t_j}^{t_{j+1}}\bi{F}|_{x_i}dt - \dfrac1\tau \displaystyle\int_{t_j}^{t_{j+1}}\bi{F}|_{x_{i-1}}dt
 	\right]=0.
 	\]
 	
 	Обозначим $\tilde{\bi{F}}(\bi{U}_i,\bi{U}_{i+1}) = \dfrac1\tau \displaystyle\int_{t_j}^{t_{j+1}}\bi{F}|_{x_i}dt$, тогда общий вид для разностных схем
 	\[
 	\dfrac{\bi{U}^{j+1}_i-\bi{U}^{j}_i}{\tau} + \dfrac1h \left[\tilde{\bi{F}}(\bi{U}_i,\bi{U}_{i+1}) -\tilde{\bi{F}}(\bi{U}_{i-1},\bi{U}_{i})\right]=0.
 	\]
 Остаётся лишь предложить способы вычисления  $\tilde{\bi{F}}(\bi{U}_i,\bi{U}_{i+1})$ для перехода к явной схеме. Метод Хартена---Лакса---ван Лира предлагает рассмотрение среднего промежуточного состояния между самой быстрой и самой медленной магнитозвуковыми волнами. %В силу структуры собственных значений матрицы $A$ выберем наименьшую и~наибольшую скорости волн и введём для них обозначения: $S_{L}=\lambda_{1}$, $S_{R}=\lambda_{7}$.% 
 
 	 Обозначим $\bi{U}_L = \bi{U}_{i-1}^{j}$, $\bi{U}_R = \bi{U}_{i}^{j}$. Пусть на границе состояний $\bi{U}_L$ и $\bi{U}_R$ минимальная скорость магнитозвуковых волн отрицательна, а максимальная --- положительна:
 	 \[S_L = \min[\lambda_1(\bi{U}_L),\lambda_1(\bi{U}_R)]<0,\]
 	 \[S_R = \max[\lambda_7(\bi{U}_L),\lambda_7(\bi{U}_R)]>0,\] то есть выполнен случай для характеристик, показанный на рис.~1.
 	%%%
 	\begin{figure}[H]
 		\centering
 	\tikzset{every picture/.style={line width=0.75pt}} %set default line width to 0.75pt        
 	\begin{tikzpicture}[x=0.75pt,y=0.75pt,yscale=-1,xscale=1]
 		%uncomment if require: \path (0,310); %set diagram left start at 0, and has height of 310
 		%Shape: Axis 2D [id:dp8901669070898124] 
 		\draw  (165,226.1) -- (530.2,226.1)(348.2,60) -- (348.2,258.1) (523.2,221.1) -- (530.2,226.1) -- (523.2,231.1) (343.2,67) -- (348.2,60) -- (353.2,67)  ;
 		%Straight Lines [id:da1741773609207251] 
 		\draw    (210.4,85.1) -- (348.2,226.1) ;
 		%Straight Lines [id:da5957105970508267] 
 		\draw    (486.4,86.1) -- (348.2,226.1) ;
 		%Straight Lines [id:da07497028391518623] 
 		\draw  [dash pattern={on 0.84pt off 2.51pt}]  (210.4,85.1) -- (211.4,228.1) ;
 		%Straight Lines [id:da15037888620942863] 
 		\draw  [dash pattern={on 0.84pt off 2.51pt}]  (210.4,85.1) -- (486.4,86.1) ;
 		%Straight Lines [id:da6121545998539613] 
 		\draw  [dash pattern={on 0.84pt off 2.51pt}]  (486.4,86.1) -- (487.4,225.1) ;
 		
 		% Text Node
 		\draw (212,65) node [anchor=north west][inner sep=0.75pt]    {$S_{L} =\frac{x}{t}$};
 		% Text Node
 		\draw (432,65) node [anchor=north west][inner sep=0.75pt]    {$S_{R} =\frac{x}{t}$};
 		% Text Node
 		\draw (517,239.4) node [anchor=north west][inner sep=0.75pt]    {$x$};
 		% Text Node
 		\draw (358,54.4) node [anchor=north west][inner sep=0.75pt]    {$t$};
 		% Text Node
 		\draw (235,174.4) node [anchor=north west][inner sep=0.75pt]    {$\bi{U}_{L}$};
 		% Text Node
 		\draw (440,174.4) node [anchor=north west][inner sep=0.75pt]    {$\bi{U}_{R}$};
 		% Text Node
 		\draw (355,101.4) node [anchor=north west][inner sep=0.75pt]    {$\bi{U}_{*}$};	
 	\end{tikzpicture}
 	\caption{Схема Риманова веера с одним промежуточным состоянием}
 \end{figure}
 	%%%
 	
 	Промежуточное состояние можно найти, проинтегрировав уравнение \eqref{vectorMHD-2} по области $(\Delta t S_L,\Delta t S_R)\times(0,\Delta t)$:
 	\[
 	\bi{U}_* = \dfrac{S_R\bi{U}_R -S_L\bi{U}_L - \tilde{\bi{F}}_R + \tilde{\bi{F}}_L}{S_R-S_L}.
 	\]
 	
 	На разрывах выполнены соотношения:
 	\[
 	\begin{cases}
 		S_L(\bi{U}_L - \bi{U}_*) = \tilde{\bi{F}}_L - \tilde{\bi{F}}_*,\\
 		S_R(\bi{U}_R - \bi{U}_*) = \tilde{\bi{F}}_R - \tilde{\bi{F}}_*.
 	\end{cases}
 	\]
 	Следовательно, можно найти выражение для $\tilde{\bi{F}}_*$:
 	\[
 		\tilde{\bi{F}}_* = \dfrac{S_R\tilde{\bi{F}}_L - S_L\tilde{\bi{F}}_R + S_RS_L(\bi{U}_R-\bi{U}_L)}{S_R-S_L}.
 	\]
 	
 	В случае же совпадения знаков скоростей магнитозвуковых волн необходимо выбрать восходящий поток. Таким образом, HLL-метод заключается в следующем:
 	\[
 	\tilde{\bi{F}}_{HLL} = 
 		\begin{cases}
 		\tilde{\bi{F}}_{L}, \phantom{xxx} S_L > 0, \\
 			\tilde{\bi{F}}_{*}, \phantom{xxx} S_L \le 0 \le S_R, \\
 			\tilde{\bi{F}}_{R}, \phantom{xxx} S_R > 0.
 		\end{cases}
 	\]
 	
 	Данный метод учитывает два основных разрыва, которые описывают распространение сильных особенностей типа ударных волн, и не рассматривает разрывы типа контактных или тангенциальных. 
 	\subsection{Метод HLLC}
 	
 	Развитием подхода HLL является схема HLLC (C - contact), учитывающая контактный (тангенциальный) разрыв, скорость которого вычисляется из решения задачи Римана в рамках ,,звукового'' приближения. 
 	
 	Предположим, что состояние $\bi{U}_*$ разделяется на два состояния $\bi{U}_{*_L}$ и  $\bi{U}_{*_R}$ контактной волной, движущейся со постоянной скоростью $S_M = u_L^* = u_R^*$.  
 	%%
 	\begin{figure}[H]
 		\centering
 	\tikzset{every picture/.style={line width=0.75pt}} %set default line width to 0.75pt        
 	\begin{tikzpicture}[x=0.75pt,y=0.75pt,yscale=-1,xscale=1]
 		%uncomment if require: \path (0,310); %set diagram left start at 0, and has height of 310
 		%Shape: Axis 2D [id:dp8901669070898124] 
 		\draw  (165,226.1) -- (530.2,226.1)(348.2,60) -- (348.2,258.1) (523.2,221.1) -- (530.2,226.1) -- (523.2,231.1) (343.2,67) -- (348.2,60) -- (353.2,67)  ;
 		%Straight Lines [id:da1741773609207251] 
 		\draw    (210.4,85.1) -- (348.2,226.1) ;
 		%Straight Lines [id:da5957105970508267] 
 		\draw    (486.4,86.1) -- (348.2,226.1) ;
 		%Straight Lines [id:da07497028391518623] 
 		\draw  [dash pattern={on 0.84pt off 2.51pt}]  (210.4,85.1) -- (211.4,228.1) ;
 		%Straight Lines [id:da15037888620942863] 
 		\draw  [dash pattern={on 0.84pt off 2.51pt}]  (210.4,85.1) -- (486.4,86.1) ;
 		%Straight Lines [id:da6121545998539613] 
 		\draw  [dash pattern={on 0.84pt off 2.51pt}]  (486.4,86.1) -- (487.4,225.1) ;
 		%Straight Lines [id:da687024001594329] 
 		\draw    (394.2,84.1) -- (348.2,226.1) ;
 		
 		% Text Node
 		\draw (205,65) node [anchor=north west][inner sep=0.75pt]    {$S_{L} =\frac{x}{t}$};
 		% Text Node
 		\draw (464,65) node [anchor=north west][inner sep=0.75pt]    {$S_{R} =\frac{x}{t}$};
 		% Text Node
 		\draw (517,239.4) node [anchor=north west][inner sep=0.75pt]    {$x$};
 		% Text Node
 		\draw (358,54.4) node [anchor=north west][inner sep=0.75pt]    {$t$};
 		% Text Node
 		\draw (235,174.4) node [anchor=north west][inner sep=0.75pt]    {$\bi{U}_{L}$};
 		% Text Node
 		\draw (440,174.4) node [anchor=north west][inner sep=0.75pt]    {$\bi{U}_{R}$};
 		% Text Node
 		\draw (295,105.4) node [anchor=north west][inner sep=0.75pt]   {$\bi{U}_{*_L}$};
 		% Text Node
 		\draw (401,106.4) node [anchor=north west][inner sep=0.75pt]   {$\bi{U}_{*_R}$};
 		% Text Node
 		\draw (384,65) node [anchor=north west][inner sep=0.75pt]    {$S_{M} =\frac{x}{t}$};
 	\end{tikzpicture}
 		\caption{Схема Риманова веера с двумя промежуточными состояниями}
 \end{figure}
 	%%
 	
 	Скорость контактной волны предлагается вычислять как $u^*$ состояния $\bi{U}_*$ в HLL, то есть
 	\[
 	S_M = \dfrac{(\rho u)^*}{\rho^*} = 
 	\dfrac{(S_R - u_R)\rho_Ru_R - (S_L - u_L)\rho_Lu_L - p_R +p_L}{(S_R - u_R)\rho_R-(S_L - u_L)\rho_L}.
 	\]
 	Далее, пользуясь условиями разрыва $S_\alpha\bi{U}_{*_\alpha} - \tilde{\bi{F}}_{*_\alpha} = S_\alpha\bi{U}_{\alpha} - \tilde{\bi{F}}_{\alpha}$, $\alpha\in\{L,R\}$, находим компоненты промежуточных состояний. 
 	
 	Однако в случае МГД-задач необходимо сделать дополнительные уточнения. Во-первых, нужно учитывать полное давление, которое должно быть непрерывным на контактной волне
 	\[
 	p_T^* = p_{T_L} + \rho_L(S_L-u_L)(S_M-u_L) = p_{T_R} + \rho_R(S_R-u_R)(S_M-u_R).
 	\]
 	Во-вторых, если компонента $B_x \not= 0$, промежуточные состояния $\bi{U}_{*_L}$ и $\bi{U}_{*_R}$ не согласуются c условием скачка на контактном разрыве, в котором тангенциальные составляющие скорости и магнитного поля должны быть непрерывными. В этом случае они заменяются средним значением HLL $\bi{U}_*$.
 	В-третьих, в данной работе был использован HLLC-L метод, который делает попытку исключить возникающие побочные осциляции, заменяя промежуточные скорости на
 	\[
 	v^*_{\alpha} = v_\alpha + \dfrac{B_x(B_{y_\alpha}-B_y^*)}{\rho_\alpha(S_\alpha-u_\alpha)}, \phantom{xxx}
 	w^*_{\alpha} = w_\alpha + \dfrac{B_x(B_{z_\alpha}-B_z^*)}{\rho_\alpha(S_\alpha-u_\alpha)}, \phantom{xxx}\alpha\in\{L,R\}.
 	\]
 	
 	Таким образом, в HLLС-методе поток вычисляется по следующей схеме:
 	\[
 	\tilde{\bi{F}}_{HLLC} = 
 	\begin{cases}
 		\tilde{\bi{F}}_{L}, \phantom{xxx} S_L > 0, \\
 		\tilde{\bi{F}}_{*_L}, \phantom{xxx} S_L \le 0 \le S_M, \\
 		\tilde{\bi{F}}_{*_R}, \phantom{xxx} S_M \le 0 \le S_R, \\
 		\tilde{\bi{F}}_{R}, \phantom{xxx} S_R > 0.
 	\end{cases}
 	\]
 	\subsection{Метод HLLD}
 	
 	Как было отмечено ранее, при решении МГД-задач методом HLLC могут возникать нефизичекие осцилляции решения, вызванные несогласованностью условий разрыва. Поэтому была предложена схема HLLD , основанная на рассмотрении четырёх промежуточных состояний $\bi{U}_{*_L}$,$\bi{U}_{**_L}$,$\bi{U}_{**_R}$, $\bi{U}_{*_R}$ на которые состояние $\bi{U}_*$ внутри Риманова веера делят энтропийная волна $S_M$ и две альфвеновские волны $S_L^*$ и $S_R^*$ \cite{Kusano}.
 	
 	\begin{figure}[H]
 		\centering
 		\tikzset{every picture/.style={line width=0.75pt}} %set default line width to 0.75pt        
 		
 		\begin{tikzpicture}[x=0.75pt,y=0.75pt,yscale=-1,xscale=1]
 			%uncomment if require: \path (0,310); %set diagram left start at 0, and has height of 310
 			
 			%Shape: Axis 2D [id:dp8901669070898124] 
 			\draw  (165,226.1) -- (530.2,226.1)(348.2,60) -- (348.2,258.1) (523.2,221.1) -- (530.2,226.1) -- (523.2,231.1) (343.2,67) -- (348.2,60) -- (353.2,67)  ;
 			%Straight Lines [id:da1741773609207251] 
 			\draw    (210.4,85.1) -- (348.2,226.1) ;
 			%Straight Lines [id:da5957105970508267] 
 			\draw    (486.4,86.1) -- (348.2,226.1) ;
 			%Straight Lines [id:da07497028391518623] 
 			\draw  [dash pattern={on 0.84pt off 2.51pt}]  (210.4,85.1) -- (211.4,228.1) ;
 			%Straight Lines [id:da15037888620942863] 
 			\draw  [dash pattern={on 0.84pt off 2.51pt}]  (210.4,85.1) -- (486.4,86.1) ;
 			%Straight Lines [id:da6121545998539613] 
 			\draw  [dash pattern={on 0.84pt off 2.51pt}]  (486.4,86.1) -- (487.4,225.1) ;
 			%Straight Lines [id:da687024001594329] 
 			\draw    (407.2,85.1) -- (348.2,226.1) ;
 			%Straight Lines [id:da42129148907773595] 
 			\draw    (293.2,86.1) -- (348.2,226.1) ;
 			%Straight Lines [id:da842465472685672] 
 			\draw    (363.2,83.1) -- (348.2,226.1) ;
 			
 			% Text Node
 			\draw (186,66.4) node [anchor=north west][inner sep=0.75pt]    {$S_{L}$};
 			% Text Node
 			\draw (484,62.4) node [anchor=north west][inner sep=0.75pt]    {$S_{R}$};
 			% Text Node
 			\draw (517,239.4) node [anchor=north west][inner sep=0.75pt]    {$x$};
 			% Text Node
 			\draw (358,54.4) node [anchor=north west][inner sep=0.75pt]    {$t$};
 			% Text Node
 			\draw (235,174.4) node [anchor=north west][inner sep=0.75pt]    {$\bi{U}_{L}$};
 			% Text Node
 			\draw (440,174.4) node [anchor=north west][inner sep=0.75pt]    {$\bi{U}_{R}$};
 			% Text Node
 			\draw (274,110.4) node [anchor=north west][inner sep=0.75pt]    {$\bi{U}_{**_{L}}$};
 			% Text Node
 			\draw (420,98.4) node [anchor=north west][inner sep=0.75pt]    {$\bi{U}_{*_{R}}$};
 			% Text Node
 			\draw (372,63.4) node [anchor=north west][inner sep=0.75pt]    {$S_{M}$};
 			% Text Node
 			\draw (411,60.4) node [anchor=north west][inner sep=0.75pt]    {$S_{R}^{*}$};
 			% Text Node
 			\draw (285,57.4) node [anchor=north west][inner sep=0.75pt]    {$S_{L}^{*}$};
 			% Text Node
 			\draw (362,93.4) node [anchor=north west][inner sep=0.75pt]    {$\bi{U}_{**_{R}}$};
 			% Text Node
 			\draw (313,98.4) node [anchor=north west][inner sep=0.75pt]    {$\bi{U}_{**_{L}}$};
 		\end{tikzpicture}
 			\caption{Схема Риманова веера с четырьмя промежуточными состояниями}
 	\end{figure}
 	
 	Как и в HLLC, $S_M$ вычисляется как $HLL$-среднее, однако вместо давлений нужно учесть полные давления:
 	\[
 		S_M = 
 	\dfrac{(S_R - u_R)\rho_Ru_R - (S_L - u_L)\rho_Lu_L - p_{T_R} +p_{T_L}}{(S_R - u_R)\rho_R-(S_L - u_L)\rho_L}.
 	\]
 	Нормальная составляющая скорости постоянна: $u_L^* = u_L^{**} = u_R^{**} = u_R^* = S_M$. Вычислим и среднее полное давление промежуточных состояний:
 	\[
 	p_T^* = \dfrac{(S_R-u_R)\rho_Rp_{T_L} - (S_L-u_L)\rho_Lp_{T_R} +\rho_L\rho_R(S_R-u_R)(S_L-u_L)(u_R-u_L) }{(S_R-u_R)\rho_R-(S_L-u_L)\rho_L}.
 	\]
 	Далее, пользуясь условиями разрыва $S_\alpha\bi{U}_{*_\alpha} - \tilde{\bi{F}}_{*_\alpha} = S_\alpha\bi{U}_{\alpha} - \tilde{\bi{F}}_{\alpha}$, $\alpha\in\{L,R\}$, находим компоненты промежуточных состояний:
 	\[
 	\rho^*_\alpha = \rho_\alpha\dfrac{S_\alpha-u_\alpha}{S_\alpha-S_M},
 	\]
 	\[
 	v_\alpha^* = v_\alpha - B_xB_{y_\alpha}\dfrac{ S_M - u_\alpha }{ \rho_\alpha(S_\alpha -u_\alpha)(S_\alpha-S_M) - B_x^2 },
 	\]
 	\[
 	w_\alpha^* = w_\alpha - B_xB_{z_\alpha}\dfrac{ S_M - u_\alpha }{ \rho_\alpha(S_\alpha -u_\alpha)(S_\alpha-S_M) - B_x^2 },
 	\]
 	\[
 	B_{y_\alpha}^* = B_{y_\alpha} \dfrac{ \rho_\alpha(S_\alpha-u_\alpha)^2-B_x^2 }{\rho_\alpha(S_\alpha -u_\alpha)(S_\alpha-S_M) - B_x^2 },
 	\]
 		\[
 	B_{z_\alpha}^* = B_{z_\alpha} \dfrac{ \rho_\alpha(S_\alpha-u_\alpha)^2-B_x^2 }{\rho_\alpha(S_\alpha -u_\alpha)(S_\alpha-S_M) - B_x^2 },
 	\]
 	\[
 	e_\alpha^* = \dfrac{ (S_\alpha-u_\alpha)e_\alpha - p_{T_\alpha}u_\alpha + p_T^*S_M + B_x(\bi{v}_\alpha \cdot \bi{B}_\alpha - \bi{v}_\alpha^* \cdot \bi{B}_\alpha^* ) }{S_\alpha-S_M}.
 	\]
 	Стоит отметить, что при $S_M = u_\alpha$, $S_\alpha = u_\alpha \pm c_{f_\alpha}$, $B_{y_\alpha}=B_{z_\alpha} = 0$ и $B_x^2 \ge \gamma p_\alpha$ могут возникать операции деления на ноль. В таких случаях нет разрыва вдоль характеристики $S_\alpha = x/t$, и следует положить
 	\[
 	v_\alpha^* = v_\alpha, \phantom{xx} w_\alpha^* = w_\alpha, \phantom{xx} B_{y_\alpha}^*=B_{z_\alpha}^* = 0, \phantom{xx} \rho_\alpha^* = \rho_\alpha, \phantom{xx} u_\alpha^* = u_\alpha, \phantom{xx} p_{T_\alpha}^* = p_{T_\alpha}.
 	\]
 	
 	Далее рассмотрим внутренние промежуточные состояния $\bi{U}_{**_\alpha}$:
 	\[
 	\rho_\alpha^{**} = \rho_\alpha^{*}, \phantom{xx} p_{T_\alpha}^{**} = p_{T_\alpha}^{*}.
 	\]
 	Скорости распространения альфвеновских волн:
 	\[
 	S_L^* = S_M - \dfrac{|B_x|}{\sqrt{\rho_L^*}}, \phantom{xxx} S_R^* = S_M + \dfrac{|B_x|}{\sqrt{\rho_R^*}}.
 	\]
 	Из условий на разрыве получаем $v_L^{**} = v_R^{**} \equiv v^{**}$, $w_L^{**} = w_R^{**} \equiv w^{**}$, $B_{y_L}^{**} = B_{y_R}^{**} \equiv B_{y}^{**}$, $B_{z_L}^{**} = B_{z_R}^{**} \equiv B_{z}^{**}$. Далее, из условия
 	\[
 	(S_R-S_R^*)\bi{U}_{*_R} + (S_R^*-S_M)\bi{U}_{**_R} + (S_M-S_L^*)\bi{U}_{**_L} + (S_L^*-S_L)\bi{U}_{*_L} -S_R\bi{U}_R + S_L\bi{U}_L + \tilde{\bi{F}}_R - \tilde{\bi{F}}_L = 0
 	\]
 	получаем выражения для соответствующих компонент:
 	\[
 	v^{**} = \dfrac{ \sqrt{\rho_L^*}v_L^* + \sqrt{\rho_R^*}v_R^* + ( B_{y_R}^{*} - B_{y_L}^{*} )\text{sign}(B_x) }{  \sqrt{\rho_L^*} + \sqrt{\rho_R^*} },
 	\]
 	\[
 	w^{**} = \dfrac{ \sqrt{\rho_L^*}w_L^* + \sqrt{\rho_R^*}w_R^* + ( B_{z_R}^{*} - B_{z_L}^{*} )\text{sign}(B_x) }{  \sqrt{\rho_L^*} + \sqrt{\rho_R^*} },
 	\]
 	\[
 	B_y^{**} = \dfrac{ \sqrt{\rho_L^*}B_{y_R}^{*} + \sqrt{\rho_R^*}B_{y_L}^{*} + \sqrt{\rho_L^*\rho_R^*}(v_R^*-v_L^*)\text{sign}(B_x) }{  \sqrt{\rho_L^*} + \sqrt{\rho_R^*} },
 	\]
 	\[
 	B_z^{**} = \dfrac{ \sqrt{\rho_L^*}B_{z_R}^{*} + \sqrt{\rho_R^*}B_{z_L}^{*} + \sqrt{\rho_L^*\rho_R^*}(w_R^*-w_L^*)\text{sign}(B_x) }{  \sqrt{\rho_L^*} + \sqrt{\rho_R^*} },
 	\]
 	\[
 	e_\alpha^{**} = e_\alpha^* \mp \sqrt{\rho_\alpha^*}(\bi{v}^*_\alpha \cdot \bi{B}^*_\alpha - \bi{v}_\alpha^{**} \cdot \bi{B}_\alpha^{**} )\text{sign}(B_x).
 	\]
 	
 	Стоит заметить, что на практике для оценки верхней и нижней границ скоростей можно использовать следующие соотношения:
 	\[
 	S_L = \min(u_L,u_R)-\max(c_{f_L},c_{f_R}), \phantom{xxx} S_R = \min(u_L,u_R) + \max(c_{f_L},c_{f_R}).
 	\]
 	
 	Подводя итог, получаем HLLD-способ вычисления МГД-потока:
 	\[
 	\tilde{\bi{F}}_{HLLD} = 
 	\begin{cases}
 		\tilde{\bi{F}}_{L}, \phantom{xxx} S_L > 0, \\
 		\tilde{\bi{F}}_{*_L}, \phantom{xxx} S_L \le 0 \le S_L^*, \\
 		\tilde{\bi{F}}_{**_L}, \phantom{xxx} S_L^* \le 0 \le S_M, \\
 		\tilde{\bi{F}}_{**_R}, \phantom{xxx} S_M^* \le 0 \le S_R^*, \\
 		\tilde{\bi{F}}_{*_R}, \phantom{xxx} S_R^* \le 0 \le S_R, \\
 		\tilde{\bi{F}}_{R}, \phantom{xxx} S_R > 0.
 	\end{cases}
 	\]
 	\section{Численные эксперименты}
 	В ходе выполнения курсовой работы представленные расчётные схемы были программно реализованы на языке C++. Визуализация полученных решений была выполнена с помощью библиотеки Matplotlib на языке Python.
 		\subsection{Распространение циркулярно поляризованной альфвеновской волны}
 		Для сравнения точности численных схем на гладких решениях рассмотрим задачу о распространении циркулярно поляризованной альфвеновской волны под углом $\alpha~=~\sfrac\pi6$ к оси $x$ в области $[0, 1/\cos\alpha]$ \cite{Lukin}. Начальные условия имеют вид:
 		\[
 		\rho = 1, \phantom{x} v_{||}=0, \phantom{x} v_{\perp} = 0.1\sin2\pi\xi,\phantom{x} w=0.1\cos2\pi\xi,
 		\]
 		\[
 		B_{||}=\sqrt{4\pi},\phantom{x} B_{\perp} = 0.1\sqrt{4\pi}\sin2\pi\xi,\phantom{x} B_z =0.1\sqrt{4\pi}\cos2\pi\xi,\phantom{x} p =0.1,
 		\]
 		где $\xi = x\cos\alpha$. Волна распространяется в направлении точки $x=0$ со скоростью $B_{||}/\sqrt{4\pi\rho}=1$. 
 		
 		Задача решим на сетке $\Omega_h^N$ с шагом $1/N$ при $N=8,16,32,64$. Для каждого расчёта оценим относительную ошибку вычислений по формуле 
 		\[
 			\delta_N(\psi) = \dfrac{\displaystyle\sum_{x\in\Omega_h^N}|\psi^N(x)-\psi^E(x)|}{\displaystyle\sum_{x\in\Omega_h^N}|\psi^E(x)|},
 		\]
 		где $\psi \in \{v_{\perp},w,B_{\perp},B_z\}$, $\psi^E$ --- точное решение (за точное решение примем решение, вычисленное при $N=128$). Скорость сходимости оценим как
 		\[
 		R_N = \log_2\dfrac{\delta_{N/2}}{\delta_N},
 		\]
 		где в качестве $\delta_N$ выбиралось среднее
 		\[
 		\delta_N = \dfrac14 (\delta_N(v_{\perp}) + \delta_N(w) + \delta_N(B_{\perp}) + \delta_N(B_z))
 		\]
 		
 		Проведём расчёт с показателем адиабаты $\gamma = 5/3$, числом Куранта $C=0.3$ до момента времени $T = 0.05$.
 		
 			\begin{figure}[H]
 			\centering
 			\hspace*{0mm}\includegraphics[width=1.\textwidth]{AlfvenWaves}
 			\caption{Задача о распространении альфвеновской волны. Ортогональная компонента $B_\perp$ магнитного поля при расчётах на различных сетках различными методам в момент времени $t=0.05$}
 		\end{figure}

 	\begin{center}
 	\begin{tabular}{|l|r|l|r|l|r|l|}
 		\hline
 		& \multicolumn{2}{c|}{HLL} & \multicolumn{2}{c|}{HLLC} & \multicolumn{2}{c|}{HLLD} \\
 		\hline
 		$N$& $\delta_N$ & $R_N$ & $\delta_N$ & $R_N$ & $\delta_N$ & $R_N$ \\
 		\hline
 		8 & 0.483645 & - & 0.481868 & - & 0.482842 & - \\
 		\hline
 		16 & 0.224146 & 1.109506 & 0.223023 & 1.111449 & 0.223643 & 1.110351 \\
 		\hline
 		32 & 0.089332 & 1.327188 & 0.088887 & 1.327153 & 0.089123 & 1.327325 \\
 		\hline
 		64 & 0.024346 & 1.875505 & 0.024202 & 1.876822 & 0.024268 & 1.876730 \\
 		\hline
 	\end{tabular}
 \vspace*{5mm}
 
 Таблица 1: Усреднённая ошибка и скорость сходимости в задаче о распространении альфвеновской волны	
 \end{center}
 	\subsection{Задача о распаде разрыва}
 	\textsl{1. Задача Брио---Ву: $x \in [-0.5, 0.5]$, $t \in [0, 0.1]$, $B_x = 0.75$.}
 	
 	При $t=0$:
 	\[
 	(\rho, p, u, v, w, B_y, B_z) = \begin{cases}
 		(1, 1, 0,0,0,1,0),  \phantom{xx}&x < 0,\\
 		(0.125, 0.1, 0,0,0,-1,0),  \phantom{xx}&x > 0.
 		\end{cases}
 	\]
 	
 	Показатель адиабаты $\gamma = 2$. Расчёт произведён с шагом $h=0.0025$ (400 узлов), числом Куранта $C=0.1$. Иллюстрация полученного решения (графики распределения плотности $\rho$, скоростей $u$ и~$v$, компонент магнитного поля $B_y$ и $B_z$, давления $p$) в~момент времени $t=0.1$ приведена на рис. 1.
 	
 	\textsl{2. Разрыв с параметрами: $x \in [-0.5, 0.5]$, $t \in [0, 0.2]$, $B_x = 4/\sqrt{4\pi}$.}
 	
 	При $t=0$:
 	\[
 	(\rho, p, u, v, w, B_y, B_z) = \begin{cases}
 		(1.08, 0.95, 1.2, 0.01, 0.5, 3.6/\sqrt{4\pi}, 2/\sqrt{4\pi}),  \phantom{xx}&x < 0,\\
 		(1, 1, 0, 0, 0, 4/\sqrt{4\pi}, 2/\sqrt{4\pi}),  \phantom{xx}&x > 0.
 	\end{cases}
 	\]
 	
 	Показатель адиабаты $\gamma = \sfrac53$. Расчёт произведён с шагом $h=0.0025$ (400 узлов), числом Куранта $C=0.2$. Иллюстрация полученного решения (графики распределения плотности $\rho$, скоростей $u$ и~$v$, компонент магнитного поля $B_y$ и $B_z$, давления $p$) в момент времени $t=0.2$ приведена на рис. 2.
 	
 	\begin{figure}[H]
 		\centering
 		\hspace*{-20mm}\includegraphics[width=1.2\textwidth]{BrioWuTest}
 		\caption{Распад разрыва в задаче Брио---Ву в момент времени $t=0.1$}
 	\end{figure}
 	
 	\begin{figure}[H]
 		\centering
 		\hspace*{-20mm}\includegraphics[width=1.2\textwidth]{ShockTubeTest}
 		\caption{Распад разрыва из теста 2 в момент времени $t=0.2$}
 	\end{figure}
 \pagebreak
 
	\section-{Заключение}
	Таким образом, в ходе выполнения курсовой работы были представлены расчётные схемы решения МГД-задачи для идеальной плазмы.
	Было проведено исследование построения методов интегро-интерполяционным методом. Реализован программный комплекс, состоящий из программы, реализующей HLL, HLLC-L и HLLD алгоритмы на языке C++, и программы, реализующей визуализацию решений на Python.
	
\begin{thebibliography}{9}
	\bibitem{Kulikovskiy} А.\,Г. Куликовский, Г.\,А. Любимов. Магнитная гидродинамика. М.: Логос, 2005. --- 328 с.
		\bibitem{Kusano} T. Miyoshi, K. Kusano. A multi-state HLL approximate Riemann solver for ideal magnetohydrodynamics. J. Comp. Phys. 208, 2005, pp. 315-344. 
		\bibitem{Lukin} М.\,П. Галанин, В.\,В. Лукин. Разностная схема для решения двумерных задач идеальной МГД на неструктурированных сетках. Препр. Инст. прикл матем. им. М.\,В. Келдыша РАН №50. 2007. 29 с. 
	\bibitem{Kulikovskiy et al} А.\,Г. Куликовский, Н.\,В. Погорелов, А.\,Ю. Семёнов. Математические вопросы численного решения гиперболических систем уравнений. М.: Наука. Физматлит. 2001. --- 608 с.
	\bibitem{Comparison} G. Mattia, A. Mignone. A comparison of approximate non-linear Riemann solvers for Relativistic MHD. MNRAS. 510, 2022, pp. 481-499. 
	\bibitem{Savenkov} М.\,П. Галанин, Е.\,Б. Савенков. Методы численного анализа математических моделей. М.: Изд-во МГТУ им. Н.Э. Баумана. 2018. --- 592 с.
	\bibitem{Ustygov}С.\,Д. Устюгов, М.\,В. Попов. Кусочно-параболический метод на локальном шаблоне. III. Одномерная идеальная МГД. Препр. Инст. прикл матем. им. М.\,В. Келдыша РАН. 2006. 27с.
\end{thebibliography}

\end{document}